\section{Role of User Stories and Backlogs in Agile Development}\label{us}
The agile software development paradigm broke the wall that classically existed between the development team and end-users. Thanks to the involvement of a Product Owner (PO) who acts as a proxy to end-users for the team, the product backlog \cite{sedano2019product} became a first-class citizen during the product development. Furthermore, thanks to a set of user stories expressing features to be implemented in the product in order to deliver value to end-users, the development teams were empowered to think in terms of added value when planning their subsequent developments. The product is then developed iteration by iteration, incrementally. Each iteration selects a subset of the stories, maintaining a link between the developers and the end-users\cite{mosser2022modelling}. 

\emph{Ordinarily, User Stories typically exhibit interdependencies, wherein the order of their implementation becomes a critical consideration. This circumstance raises the question of how one may discern and identify the relationships that exist between the User Stories.}

Sedano et al. posited that a “product backlog is an informal model of the work to be done” \cite{sedano2019product}. A backlog implements a shared mental model over practitioners working on a given product, acting as a boundary artifact between stakeholders. This model is voluntarily kept informal to support rapid prototyping and brainstorming sessions. Classically, backlogs are stored in project management systems, such as \footnote{\href{https://www.atlassian.com/en/software/jira}{Jira Software}} Jira . These tools stores user stories as tickets, where stakeholders write text as natural language. Meta-data (e.g., architecture components, severity, quality attribute) can also be attached to the stories. However, there is no formal language to express stories or model backlogs from a state of practice point of view.

A user story is a brief, semi-structured sentence and informal description of some aspect of a software system that illustrates requirements from the user’s perspective \cite{raharjana2021user}. Large, vague stories are called epics. While user stories vary widely between organizations, most observed stories included a motivation and acceptance criteria. The brief motivation statement followed the pattern:  As a  \textless\emph{user}\textgreater\ I want to \textless\emph{action}\textgreater\ so that \textless\emph{value}\textgreater. This is sometimes called the Connextra template . The acceptance criteria followed the pattern: Given \textless\emph{context}\textgreater, when \textless\emph{condition}\textgreater \  then \textless\emph{action}\textgreater. This is sometimes called Gherkin syntax \cite{wynne2017cucumber}. It consists of three aspects, namely aspects of who, what and why. the aspect of “who” refers to the system user or actor, “what” refers to the actor’s desire, and “why” refers to the reason (optional in the user story) \cite{raharjana2021user}.

The user story components consist of the following elements \cite{wautelet2017user} : \emph{Role}: abstract behavior of actors in the system context; the aspect of who representation. \emph{Goal}: a condition or a circumstance desired by stakeholders or actors. Task: specific things that must be done to achieve goals. \emph{Capability}: the ability of actors to achieve goals based on certain conditions and events.
