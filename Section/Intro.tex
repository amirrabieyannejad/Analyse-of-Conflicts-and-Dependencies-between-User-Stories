\section{Introduction}
User stories, the fundamental building blocks of software development, serve as concise and testable descriptions of a program's functionality. Within the dynamic framework of agile development, these user stories are typically composed informally, using plain text, and they are maintained in the product backlog, which acts as a repository for prioritizing and tracking development tasks.

Behavior Driven Development (BDD), a specialized approach within the realm of agile software development, places a strong emphasis on the iterative implementation of user stories. The sequencing of user stories in BDD is a pivotal aspect of the methodology. The right sequence not only impacts the efficiency of development but also the overall success of the project. By prioritizing and sequencing user stories effectively, development teams can deliver incremental value to users, respond to changing requirements, and ensure that the most critical functionality is addressed first.

User stories often exhibit dependencies on one another, leading to potential conflicts in which one user story necessitates the deletion of a component vital to the successful execution of another user story, or one user story may introduce an element that contravenes and thus prohibits the realization of another user story. To minimize the occurrence of conflicts, teams should systematically identify and document dependencies between user stories within their backlog. Agile methodologies, such as Scrum, promote cross-functional collaboration and daily stand-up meetings as mechanisms for promptly addressing and mitigating dependencies and conflicts. However, this approach can be time and resource-intensive. In instances where the backlog is extensive, the recognition of existing conflicts between user stories can become a complex endeavor.

To achieve the automation of conflict and dependency resolution within the scope of user stories associated with a single backlog, it is imperative that we establish a well-structured workflow. This workflow should encompass a collection of techniques and tools derived from various domains. 

In this research paper, we embark on a comprehensive exploration of cutting-edge techniques and methodologies in the realm of natural language processing (NLP) and computational lexicon resources, a symbiotic relationship that holds pivotal significance for tasks pertaining to natural language comprehension. Furthermore, we embark on an exploration of techniques which extracting domain models from textual requirements, pivotal disciplines enabling the application of graph transformation rules and the discernment of conflicts and dependencies among user stories.

Section \ref{usq} introduces fundamental concepts, providing the necessary background information about \emph{user story} and \emph{backlog}. Additionally, we delve into the techniques employed in the User Story Quality and compare these techniques with each other. Moving forward to Section \ref{dmodel}, we conduct a comparative analysis of various approaches extracting domain models from textual requirements with a specific focus on efficient backlog management geared towards the recognition of user stories. Section \ref{nlp} is dedicated to a comparative analysis of several computational lexical resource techniques. The aim is to identify the most appropriate verb lexicon for the categorization of verbs found within user stories into three distinct categories, thereby facilitating the formulation of precise transformation rules. Section \ref{gts} centers on the comparison of methods for generating graph transformation rules. Finally, section \ref{conclusion} conclude the paper. 
