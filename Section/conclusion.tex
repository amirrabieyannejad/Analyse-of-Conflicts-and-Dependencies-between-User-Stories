\section{Conclusion}\label{Conclusion}
In section \ref{usq}, we conducted a comparative analysis of different techniques for assessing the quality of user stories, categorized according to different criteria. The INVEST criteria are considered applicable in manual environments, such as those monitored by product owners, and require manual assessment against these criteria. The QUS framework, on the other hand, was implemented through a tool called AQUSA to automate the process of assessing the quality of user stories. In the initial phase of our workflow, we use the QUS framework and AQUSA as tools to examine user stories within the backlog and ensure their applicability and alignment for subsequent actions.

In the \ref{dmodel} section, we conducted an experiment to evaluate the performance of the Visual Narrator, GPT 3.5 and CRF-based approach in automating the extraction of domain concepts from Agile product backlogs. Since the CRF generates a graph-based model, it is particularly beneficial to our approach as it serves as input for the development of a transformation rule system.

Section \ref{nlp} is about comparing different lexical resource techniques for computation. VerbNet, which specializes in verbs, FrameNet, which includes a wider range of nouns and adjectives, and WordNet, which provides a large selection of words covering different parts of speech, were evaluated.

For our purposes, VerbNet is the most suitable technique. The hierarchical classification of verbs into classes provides a structured and comprehensive approach to categorizing a wide range of verbs based on their semantics. This is of utmost importance in our effort to formulate transformation rules based on semantic interpretations of actions within user stories.

Finally, in section \ref{gts} we examine the graph transformation tools, particularly Henshin and GROOVE. GROOVE is best suited for comprehensive analysis of the entire state space and random-linear exploration and proves to be extremely effective for analysis and verification purposes. Henshin, on the other hand, is an EMF-based transformation model whose main features are scalability and interoperability.

In particular, Henshin supports conflict and dependency analysis using the CPA extension. Another compelling feature of Henshin lies in the intrinsic provision of a versatile application programming interface (API) through the CPA extension.

Due to the use of various tools and techniques to identify dependencies and conflicts between user stories, several relevant questions arise: Can we manage to roughly divide all verbs into three categories, namely \enquote{create}, \enquote{delete} and \enquote{forbid}? Regarding the quality analysis presented, there are not suitable analyzes for all criteria. What could the analysis look like for missing criteria? Can we use graph transformations to find conflicts and dependencies between annotated user stories? To what extent does the effectiveness of our approach depend on the data sets provided in the backlogs? Would we characterize the overall effectiveness of our approach?