\section{User Story Quality Analysis Techniques}\label{usq}
In this section, we embark on an exposition of methodologies with the specific objective of refining the precision of User Stories measurement. First, in subsection \ref{us} the role of USs and backlogs in agile development will be discussed. Additionally, the most common pattern of US will be introduced.

From subsection \ref{invest} our focus centers on the exploration of established techniques to address the query regarding the existence of criteria for the management and identification of conflicts and dependencies among User Stories. Furthermore, our attention is directed towards the criteria they employ, a pivotal factor in ensuring the creation of well-structured and standardized User Stories. After this exploration, we undertake a comprehensive comparative analysis of these methodologies to ascertain the approach that best aligns with our particular contextual requirements.
\subsection{Role of User Stories and Backlogs in Agile Development} \label{us}
The agile software development paradigm broke the wall that classically existed between the development team and end-users. Thanks to the involvement of a Product Owner (PO) who acts as a proxy to end-users for the team, the product backlog \cite{sedano2019product} became a first-class citizen during the product development. Furthermore, thanks to a set of user stories expressing features to be implemented in the product in order to deliver value to end-users, the development teams were empowered to think in terms of added value when planning their subsequent developments. The product is then developed iteration by iteration, incrementally. Each iteration selects a subset of the stories, maintaining a link between the developers and the end-users\cite{mosser2022modelling}. 

\textbf{\emph{Ordinarily, User Stories typically exhibit interdependencies, wherein the order of their implementation becomes a critical consideration. This circumstance raises the question of how one may discern and identify the relationships that exist between the User Stories.}}

Sedano et al. posited that a “product backlog is an informal model of the work to be done” \cite{sedano2019product}. A backlog implements a shared mental model over practitioners working on a given product, acting as a boundary artifact between stakeholders. This model is voluntarily kept informal to support rapid prototyping and brainstorming sessions. Classically, backlogs are stored in project management systems, such as \footnote{\href{https://www.atlassian.com/en/software/jira}{Jira Software}} Jira. These tools store user stories as tickets, where stakeholders write text as natural language. Meta-data (e.g., architecture components, severity, quality attribute) can also be attached to the stories. However, there is no formal language to express stories or model backlogs from a state of practice point of view.

A user story is a brief, semi-structured sentence and informal description of some aspect of a software system that illustrates requirements from the user’s perspective \cite{raharjana2021user}. Large, vague stories are called epics. While user stories vary widely between organizations, most observed stories included a motivation and acceptance criteria. The brief motivation statement followed the pattern:  As a  \textless\emph{user}\textgreater\ I want to \textless\emph{action}\textgreater\ so that \textless\emph{value}\textgreater. This is sometimes called the Connextra template. The acceptance criteria followed the pattern: Given \textless\emph{context}\textgreater, when \textless\emph{condition}\textgreater \  then \textless\emph{action}\textgreater. This is sometimes called Gherkin syntax \cite{wynne2017cucumber}. It consists of three aspects, namely aspects of who, what and why. The aspect of “who” refers to the system user or actor, “what” refers to the actor’s desire, and “why” refers to the reason (optional in the user story) \cite{raharjana2021user}.

The user story components consist of the following elements \cite{wautelet2017user} : \emph{Role}: abstract behaviour of actors in the system context; the aspect of who representation. \emph{Goal}: a condition or a circumstance desired by stakeholders or actors. Task: specific things that must be done to achieve goals. \emph{Capability}: the ability of actors to achieve goals based on certain conditions and events.
\subsection{Improving User Story using INVEST Criteria} \label{invest}
In Agile software development, effective requirement management is crucial for delivering valuable and high-quality software. The INVEST criteria serve as a set of guiding principles that help teams assess and shape their user stories and requirements to meet the demands of Agile development. Each letter in the acronym INVEST represents a fundamental attribute that a requirement should possess to maximize its effectiveness within an Agile context.

These criteria were introduced to ensure that Agile teams create requirements that are not only clear and actionable but also adaptable to changing circumstances. The goal is to promote flexibility, collaboration, and a relentless focus on delivering value to the end-users or stakeholders \cite{buglione2013improving}.

In the following, we will examine the individual INVEST criteria in depth, with a focus on explaining their importance and their pragmatic application in the context of Agile Requirement Engineering \cite{cohn2004user}. \\ 
\textbf{Independent} \\ 
Independent emphasize the avoidance of interdependencies between user stories or requirements. This principle underscores the importance of ensuring that each requirement operates autonomously, without relying on the completion of other related requirements. The presence of dependencies among requirements can introduce complexities in the prioritization and planning phases of Agile development. Such dependencies may necessitate a specific order of implementation, hindering the team's ability to adapt to changing priorities and potentially causing delays in project delivery. \\ 
\textbf{Negotiable}\\ 
 USs (User Stories) should not be construed as rigid, contractual obligations or exhaustive lists of requirements that the software must adhere to. Instead, they are viewed as dynamic entities, open to ongoing discussion and refinement.

US are not meant to encapsulate every conceivable detail at the outset. The inclusion of excessive details can create a deceptive illusion of precision or completeness, potentially stifling the need for continued dialogue and collaboration. 

An essential attribute of negotiable USs are their inherent flexibility. It is crucial to avoid treating User Stories as contractual obligations, as this can lead to a rigid mindset where estimates and commitments are made with the same level of inflexibility as a traditional contract. Instead, Agile teams should recognize the negotiable nature of USs and remain open to adapting them as circumstances warrant. \\ 
\textbf{Valuable}\\ 
The criterion of Value underscores that US must possess inherent worth and significance. A US should be designed in such a way that it delivers tangible value to either the end-users or the stakeholders. The value encapsulated within a US can manifest in various forms, such as improved user experiences, enhanced functionality, increased efficiency, or meeting specific business objectives.

Agile teams should prioritize and select USs that offer real, quantifiable value, as this ensures that the development effort is consistently directed toward outcomes that align with the project's overarching goals and objectives.

The concept of Value serves as a guiding principle that encourages Agile teams to continuously assess and prioritize USs based on their potential to contribute positively to the project's success, ultimately enhancing the overall quality and impact of the software being developed.\\ 
\textbf{Estimable}\\ 
An essential condition for a US to be deemed Estimable is that it must be sufficiently well-defined and clear. Ambiguity or lack of clarity within a US can render it inestimable, meaning that the development team cannot provide a reliable estimate of the effort required for implementation.

Estimable USs facilitate the planning process by enabling the team to make informed decisions about the allocation of resources, timeframes, and priorities. Clear and concise USs ensure that the estimation process is based on a solid foundation of understanding, leading to more accurate project planning and execution. \\  
\textbf{Small}\\ 
The attribute of smallness is a pivotal aspect of the INVEST criteria for USs. It pertains to the size and granularity of individual USs. Large stories, present several challenges in Agile development. They tend to be intricate and challenging to estimate accurately. 

A Complex Story refers to a US that is inherently large and resistant to disaggregation into smaller, more granular stories. Complex stories are a source of concern in Agile development because they can hinder effective planning and may require special consideration and strategies to address. Additionally, they pose difficulties in terms of fitting into single iterations or sprints.\\ 
\textbf{Testable}\\ 
The testable attribute signifies that USs should be structured in a manner that allows for the clear demonstration of whether they meet the customer's expectations. The verification of meeting these expectations is typically accomplished through testing.

An overarching goal in Agile development is to maximize the efficiency and effectiveness of testing processes. To achieve this, Agile teams strive for a high degree of test automation, with a target of attaining a test automation rate exceeding 90\%.

Test automation plays a pivotal role in Agile development by enabling rapid and repeatable testing of software features. Automated tests ensure that the software remains consistent with the desired functionality outlined in the US, thereby reducing the risk of regressions and defects.

INVEST grid in table \ref{tb:invest} can be used as a template when the customer and provider meet to evaluate a US \cite{buglione2013improving}.

The INVEST criteria serve as fundamental guidelines that are partially incorporated into the definition of many existing quality frameworks and requirement development tools in Agile methodologies.\\ \\ 


\newgeometry{margin=2.5cm}

\thispagestyle{empty}


\begin{figure}

\begingroup

\footnotesize
\Rotatebox{90}{%

\begin{tabularx}{25cm}{c  X  X  X  X  X}
 INVEST & \multicolumn{1}{c}{Description} & \multicolumn{1}{c}{0} & \multicolumn{1}{c}{1} & \multicolumn{1}{c}{2} & \multicolumn{1}{c}{3} \\
  &  & \multicolumn{1}{c}{Poor/Absent} & \multicolumn{1}{c}{Fair} & \multicolumn{1}{c}{Good} & \multicolumn{1}{c}{Excellent} \\
\hline
\hline
\\
 I – Independent & User Stories should be as independent as possible & The start of construction of a US is tied to the completion of at least one other US & The completion of a US hinders the start of construction of at least one other US  & The US can contain any constraint, but its release can be constrained by the completion of at least one other US  & The US is fully independent, and it can be realized and released with any constraint \\
 \\
 N – Negotiable & User Stories should be \enquote{open}, reporting any relevant details as much as possible & The US contains enough detail to be a technical specification (Design phase), leaving no room to negotiate any element & The US is written with enough detail to be a functional specification (Analysis phase), leaving no room to negotiate any element & The US is written with informative content defining a User Requirement in a consolidated manner, yet shared between Customer and Provider  & The US is written with the informative content typical of a high-level need, allowing feedback between customer and provider \\
\\
 V – Valuable & User Stories should provide value to end users in terms of the solution & The functional part (F) of the US does not contain all the functionalities requested by the customer & The functional (F) part of the US expresses mostly qualitative (Q) and technical (T) requirements about the system, and needs to be more developed in terms of functional requirements & The functional (F) part of the US expresses mostly the functional requirements requested by the Customer, but also includes qualitative (Q) and technical (T) requirements & The functional (F) part of the US correctly expresses only the functional requirements requested by the customer \\

 \\
 E – Estimable & Each User Story must be able to be estimated in terms of relative size and effort & The US shows only its functional (F) part, filled in by the customer, but without sufficient detail to allow the provider to fill in the Q/T parts & The US shows only its functional (F) part, filled in by the customer, but validated with the provider  & The US has been completed by the provider with respect to Q/T issues, but still needs to be validated jointly with the customer & All the useful parts of the US (F/Q/T) are shown, allowing the effort need to size and estimate it, and validated by both parts \\

 \\
 S – Small & Each User Story should be sufficiently granular, and not defined at too high a level & The US is very large, and cannot be completed within a Sprint & The US is very large, and can be completed within a Sprint, but cannot accommodate the creation/delivery of other US & The size of the US is such that it can be completed within a Sprint, jointly with other US, but it is too small to create overhead about the Testing phase  & The size of the US is such that it can be completed within a Sprint, jointly with other US, ensuring an appropriate balance between development and testing activities \\

 \\
 T – Testable & Each User Story must be formulated in an effort to stress useful details for creating tests & The US does not include tips about Acceptance Tests & The US includes a formal indication of Acceptance Tests, but yet to be completed & The US includes an indication of Acceptance Tests which are complete, but yet to be validated & The US includes an indication of completed and validated Acceptance Tests \\
 \\
 \hline

\end{tabularx}

}%
\begin{TableCaption}
\caption{INVEST Grid \cite{buglione2013improving}}\label{tb:invest}
\end{TableCaption}
\endgroup
\end{figure}
\restoregeometry
\subsection{A Quality Framework} \label{usq_2}
In this section we introduce the quality framework which use traditional requirement known as the Software Product Certification Model (SPCM)\cite{heck2010software} to establish quality criteria for agile requirements. SPCM is based on extensive literature research for traditional up-front requirements engineering. 

In order to certification two types of input are required: (1) one or more software artifacts and (2) on or more properties of these artefacts that are to be certified. The SPCM divides a software artifact into six Product Areas, namely the \emph{Context Description}, which describes the environment of the system, the \emph{User Requirement}, the \emph{High-Level Design}, the \emph{Detailed Design}, the \emph{Implementation} and \emph{Test}. Next to the division, the SPCM defines specific certification criteria for each area \cite{heck2010software}. The properties of these artifacts which Heck et al. have denoted as \enquote{Conformance Properties}, can fall into one of the following categories:

\begin{itemize}
\item Consistency: do the different (parts of) software artifacts conform to each other?
\item Functional: does input to the system produce the expected output?
\item Behavioral: does the system meet general safety and progress properties like absence of deadlocks or are constraints on the specific states of the system met?
\item Quality: do the artifacts fulfill nonfunctional requirements in the areas of for example performance, security, and usability?
\item	Compliance: do the artifacts conform to standards, guidelines, or legislation?

\end{itemize}

Based on the SPCM they define three overall criteria for agile requirements \cite{heck2014quality}: 
\textbf{Completeness}\\ 
All elements of the agile requirement should be present. Three levels have considered: \emph{basic} elements, \emph{required} elements, \emph{optional }elements. In that way it can differentiate between elements that are absolutely mandatory for a requirement and elements that are nice to have because they increase the requirement quality. \\ 
\textbf{Uniformity}\\ 
The style and format of the agile requirements should be standardized, because this leads to less time for understanding and managing the requirements. Each time a team member is confronted with a new requirement he/she needs some time to understand the requirement and decide what to do with it. This process takes less time when the requirements format is standardized. Then all team members know where to look for what information on the requirement or how to read certain models attached to the requirement. \\ 
\textbf{Conformance}\\ 
Each element in the requirements is described in a correct and consistent way. The relations between the elements in the requirements description and with the context description are correct and consistent. They should be subject to manual verification, as: two requirements or use cases contradict each other; No requirement is ambiguous; Functional requirements specify what, not how; Each requirement is testable; Each requirement is uniquely identified; Each requirement is atomic; The glossary definitions are non-cyclic; Use case diagrams correspond to use case text; The data model diagram is in normal form.

The following criteria are explicitly delineated for USs, as depicted in figure \ref{fig:arvf}:
\begin{figure}
\center
\includegraphics[width=13.03cm, height=7.76cm]{Agile_Requirements_Verification_Framework}
\caption{Agile Requirements Verification Framework \cite{heck2014quality}}\label{fig:arvf}
\end{figure}


\begin{itemize}
\item Basic Elements: Role, activity, business value (‘Who needs what why?’) instead of summary and description
\item Required Elements: acceptance criteria or acceptance tests to verify the story instead of rationale
\item Optional Elements: the team could agree to more detailed attachments to certain user stories (e.g. UML models) for higher quality
\item Stories Uniform: each user story follows the standard user voice form
\item Attachments Uniform: any modeling language used in the attachments is uniform and standardized
\end{itemize}


\subsection{QUS framework} \label{usq_3}
Lucassen et al. \cite{lucassen2016improving} represent a Quality User Story (QUS) framework, which consist of 13 quality criteria that US writers should strive to conform to. Subjected criteria determine the intrinsic quality of USs in terms of syntax, pragmatics, and semantics (Figure \ref{fig:qus_framework}; Table \ref{tb:qus}). Base on QUS, Lucassen et al. present the Automatic Quality User Story Artisan (AQUSA) software tool for assessing and enhancing US quality. Relying on NLP techniques, AQUSA detects quality defects and suggests possible remedies.

A user story should follow some pre-defined, agreed upon template chosen from the many existing ones \cite{wautelet2014unifying}. The skeleton of the template is called \emph{format} in the conceptual model, in between which the \emph{role}, \emph{means}, and optional \emph{end(s)} are interspersed to form a user story. 

Because USs are a controlled language, the QUS framework’s criteria are organize in Lindland’s categories \cite{lindland1994understanding}:

\begin{itemize}
\item\emph{ Syntactic quality}, concerning the textual structure of a US without considering its meaning;
\item \emph{Semantic quality}, concerning the relations and meaning of (parts of) the US text;
\item \emph{Pragmatic quality}, considers the audience’s subjective interpretation of the user story text aside from syntax and semantics.
\end{itemize}

First, Lucassen et al. introduced quality criteria that can be evaluated against an individual US by presenting an explanation of the criterion as well as an example US that violates the specific criterion.


\begin{figure}
\center
\includegraphics[width=10.03cm, height=7.76cm]{Quality_US_framework_that_define_13_criteria_for_US_quality_overview}
\caption{Agile Requirements Verification Framework \cite{lucassen2016improving}}\label{fig:qus_framework}
\end{figure}

\begin{figure}
\begingroup
\footnotesize

\begin{tabularx}{\textwidth}{l  X  c}
 \hline
 Criteria & Description & Individual/Set \\
\hline
\hline
\\
\textbf{Syntactic} \\
\\ 
Well-formed & A user story includes at least a role and a means & Individual \\
Atomic & A user story expresses a requirement for exactly one feature & Individual \\
Minimal & A user story contains nothing more than role, means, and ends & Individual \\
\\ 
 \textbf{Semantic} \\
 \\ 
Conceptually sound&The means expresses a feature and the ends expresses a rationale & Individual\\
Problem-oriented& A user story only specifies the problem, not the solution to it& Individual\\
Unambiguous&A user story avoids terms or abstractions that lead to multiple interpretations &Individual \\
Conflict-free&A user story should not be inconsistent with any other user story &Set \\
\\ 
\textbf{Pragmatic}\\
\\ 
Full sentence&A user story is a well-formed full sentence &Individual \\
Estimable&A story does not denote a coarse-grained requirement that is difficult to plan and prioritize &Individual \\
Unique&Every user story is unique, duplicates are avoided &Set \\
Uniform&All user stories in a specification employ the same template &Set \\
Independent&The user story is self-contained and has no inherent dependencies on other stories &Set \\
Complete&Implementing a set of user stories creates a feature-complete application, no steps are missing &Set \\
 \\
 \hline

\end{tabularx}
\begin{TableCaption}
\caption{Quality User Story framework that defines 13 criteria for user story quality: details \cite{lucassen2016improving}}\label{tb:qus}
\end{TableCaption}
\endgroup
\end{figure}
\textbf{Independent}\\ 
USs should not overlap in concept and should be schedule and implementable in any order. 

Complete independence may not always be achievable, the recommendation is to make any dependencies explicit and visible. Additionally, resolving certain dependencies may not be possible, and it is suggested practically approaches such as adding notes to story cards or using hyperlinks in issue trackers to make these dependencies evident. Two illustrative cases of dependencies are presented:
\begin{itemize}
\item 	\emph{Causality}: In some cases, one user story ($l_1$) must be completed before another ($l_2$) can begin. This is formalized as the predicate \enquote{$hasDep(l_1, l_2)$}, indicating that $ l_1$ causally depends on $l_2$ when specific conditions are met.
\item 	\emph{Superclasses}: USs may involve an object (\emph{e.g.}, \enquote{Content} in US \enquote{As a User, I am able to edit the content that I added to a person’s profile page}) that refers to multiple other objects in various stories, implying that the object in $l_1$ serves as a parent or superclass for the other objects.
\end{itemize}
\textbf{Well-formed}\\ 
Before it can be considered a US, the core text of the requirement needs to include a role and the expected functionality: the \emph{means}. Considering the US \enquote{I want to see an error when I cannot see recommendations after I upload an article}. It is likely that the US writer has forgotten to include the role. The story can be fixed by adding the role: \enquote{As a Member, I want to see an error when I cannot see recommendations after I upload an article.}.\\ 
\textbf{Atomic}\\ 
A user story should concern only one feature. Although common in practice, merging multiple user stories into a larger, generic one diminishes the accuracy of effort estimation\cite{liskin2014we}. For instance, the US \enquote{As a User, I am able to click a particular location from the map and thereby perform a search of landmarks associated with that latitude longitude combination} consist of two separate requirements: the act of clicking on a location and the display of associated landmarks. This US should be split into two:
\begin{itemize}
\item $US_A$: \enquote{As a User, I’m able to click a particular location from the map};
\item $US_B$: \enquote{as a User, I’m able to see landmarks associated with the latitude and longitude combination of a particular location}.
\end{itemize}
\textbf{Minimal}\\ 
User stories should contain a role, a means, and (optimally) some ends. Any additional information such as comments, descriptions of the expected behaviour, or testing hints should be left to additional notes. Consider the US \enquote{As a care professional, I want to see the registered hours of this week (split into products and activities). See: Mockup from Alice NOTE—first create the overview screen—then add validations}: Aside from a role and means, it includes a reference to an undefined mockup and a note on how to approach the implementation. The requirements engineer should move both to separate user story attributes like the description or comments, and retain only the basic text of the story: \enquote{As a care professional, I want to see the registered hours of this week.} \\ 
\textbf{Conceptually sound}\\ 
The means and end parts of a user story play a specific role. The means should capture a concrete feature, while the end expresses the rationale for that feature. Consider the US \enquote{As a User, I want to open the interactive map, so that I can see the location of landmarks}: The end is actually a dependency on another (hidden) functionality, which is required in order for the means to be realized, implying the existence of a landmark database which is not mentioned in any of the other stories. A significant additional feature that is erroneously represented as an end, but should be a means in a separate user story, for example:
\begin{itemize}
\item $US_A$: \enquote{As a User, I want to open the interactive map};
\item $US_B$: \enquote{As a User, I want to see the location of landmarks on the interactive map.}.
\end{itemize}
\textbf{Problem-oriented}\\ 
In line with the problem specification principle for RE proposed by Zave and Jackson \cite{zave1997four}, a user story should specify only the problem. If absolutely necessary, implementation hints can be included as comments or descriptions. Aside from breaking the minimal quality criteria, this US \enquote{As a care professional I want to save a reimbursement—add save button on top right (never grayed out)} includes implementation details (a solution) within the user story text. The story could be rewritten as follows: \enquote{As a care professional, I want to save a reimbursement.}. \\ 
\textbf{Unambiguous}\\ 
Ambiguity is intrinsic to natural language requirements, but the requirements engineer writing user stories has to avoid it to the extent this is possible. Not only should a user story be internally unambiguous, but it should also be clear in relationship to all other user stories. The Taxonomy of Ambiguity Types \cite{berry2004ambiguity} is a comprehensive overview of the kinds of ambiguity that can be encountered in a systematic requirements specification.

In this US \enquote{As a User, I am able to edit the content that I added to a person's profile page}, \enquote{content} is a superclass referring to audio, video, and textual media uploaded to the profile page as specified in three other, separate user stories in the real-world user story set. The requirements engineer should explicitly mention which media are editable; for example, the story can be modified as follows: \enquote{As a User, I am able to edit video, photo and audio content that I added to a person’s profile page.}. \\ 
\textbf{Full sentence}\\ 
A user story should read like a full sentence, without typos or grammatical errors. For instance, the US \enquote{Server configuration} is not expressed as a full sentence (in addition to not complying with syntactic quality). By reformulating the feature as a full sentence user story, it will automatically specify what exactly needs to be configured. For example, US \enquote{Server configuration} can be modified to \enquote{As an Administrator, I want to configure the server’s sudo-ers.} \\ 
\textbf{Estimatable}\\ 
As user stories grow in size and complexity, it becomes more difficult to accurately estimate the required effort. Therefore, each user story should not become so large that estimating and planning it with reasonable certainty becomes impossible \footnote{\href{http://xp123.com/articles/invest-in-good-stories-and-smart-tasks/. Accessed 2015-02-18}{INVEST in good stories, and SMART tasks}}. For example, the US \enquote{As a care professional I want to see my route list for next/future days, so that I can prepare myself (for example I can see at what time I should start traveling)} requests a route list so that care professionals can prepare themselves. 

While this might be just an unordered list of places to go to during a workday, it is just as likely that the feature includes ordering the routes algorithmically to minimize distance travelled and/or showing the route on a map. These many functionalities inhibit accurate estimation and call for splitting the user story into multiple user stories; for example:
\begin{itemize}
\item $US_A$: \enquote{As a Care Professional, I want to see my route list for next/future days, so that I can prepare myself};
\item $US_B$: \enquote{As a Manager, I want to upload a route list for care professionals.}.
\end{itemize}

The subsequent quality criteria pertain to a collection of USs. These quality criteria are instrumental in the assessment of the overall project specification's quality, focusing on the entirety of the project specification as opposed to the individual scrutiny of individual stories: \\ 
\textbf{Unique and Conflict-Free}\\ 
The concept of unique user stories, emphasizing the avoidance of semantic similarity or duplication within a project. For example considering $EP_a$: \enquote{as a Visitor, I am able to see a list of news items, so that I stay up to date} and $US_a$: \enquote{As a Visitor, I am able to see a list of news items, so that I stay up to date}. This situation can be improved by providing more specific stories, like:
\begin{itemize}
\item $US_{\text{a1}}$ \enquote{As a Visitor, I am able to see breaking news;}
\item $US_{\text{a2}}$ \enquote{As a Visitor, I am able to see sports news.}
\end{itemize}
Additionally, the importance of avoiding conflicts between user stories should be considered to ensure their quality. \textbf{\emph{A requirements conflict occurs when two or more requirements cause an inconsistency}} \cite{paja2013managing} \cite{robinson1989integrating}. For instance, considering story $US_b$: \enquote{As a User, I am able to edit any landmark} contradicts the requirement that a user can edit any landmark ($US_c$: \enquote{As a User, I am able to delete only the landmarks that I added}), if we assume that editing is a general term that includes deletion too. $US_b$ refers to any landmark, while  $US_c$ only those that user has added. A possible way to fix this is to change $US_b$ to: \enquote{As a User, I am able to edit the landmarks that I added.} \cite{lucassen2016improving}

%For instance, considering the stories $US_b$:\enquote{As a User, I am able to edit any landmark} and $US_c$: \enquote{As a User, I am able to delete only the landmarks that I added} and assuming that editing is a general term that includes deletion, these two user stories are contradicting. The conflict is \enquote{any landmark} versus \enquote{the landmark that I added}.  A possible way to fix this is to delete one of the user stories or explicitly excluding the deletion from $US_b$ (\emph{i.e.} \enquote{As a User, I am able to add and modify any landmark})
To detect these types of relationships, each US part needs to be compared with the parts of the other USs, using a combination of similarity measures that are either syntactic (\emph{e.g.}, Levenshtein’s distance) or semantic (\emph{e.g.}, employing an ontology to determine synonyms). When similarity exceeds a certain threshold, a human analyst is required to examine the user stories for potential conflict and/or duplication.
\begin{definition}
A user story $\mu$ is a 4-tuplel $\mu=(r,m,E,f)$ where $r$ is the role, $m$ is the means, $E=(e_1, e_2, . . .)$ is a set of ends, and $f$ is the format. A means m is a 5-tuple $m (s,av,do,io,adj)$ where $s$ is a subject, $av$ is an action verb, $do$ is a direct object, $io$ is an indirect object, and $adj$ is an adjective (io and adj may be null, see Figure \ref{fig:conceptual_model}). The set of user stories in a project is denoted by $U=(\mu_1, \mu_2, . . .)$.
\end{definition}
\begin{figure}
\center
\includegraphics[width=10.03cm, height=7.76cm]{Conceptual_model_of_US}
\caption{Conceptual model of user stories \cite{lucassen2016improving}}\label{fig:conceptual_model}
\end{figure}
\begin{definition}
\emph{Different means, same end }Two or more user stories that have the same end, but achieve this using different means. This relationship potentially impacts two quality criteria, as it may indicate: (1) a feature variation that should be explicitly noted in the user story to maintain an unambiguous set of user stories, or (2) a conflict in how to achieve this end, meaning one of the user stories should be dropped to ensure conflict-free user stories. Formally, for user stories $\mu_1$ and $\mu_2$:\\ 
$diffMeansSameEnd(\mu_1,\mu_2)\leftrightarrow m_1 \neq m_2 \wedge E_1 \cap E_2 \neq \emptyset$
\end{definition}
\begin{definition}
\emph{Same means, different end} Two or more user stories that use the same means to reach different ends. This relationship could affect the qualities of user stories to be unique or independent of each other. If the ends are not conflicting, they could be combined into a single larger user story; otherwise, they are multiple viewpoints that should be resolved. Formally,\\ 
$sameMeansDiffEnd(\mu_1, \mu_2) \leftrightarrow m_1 = m_2 \wedge (E_1 \setminus E_2 \neq \emptyset \vee E_2 \setminus E_1 \neq \emptyset )$
\end{definition}
\begin{definition}
\emph{Full duplicate} $A$ user story $\mu_1$ is an exact duplicate of another user story  $\mu_2$ when the stories are identical. This impacts the unique quality criterion. Formally,\\ 
$isFullDuplicate(\mu_1,\mu_2) \leftrightarrow \mu_1 =_{\text{syn}} \mu_2$
\end{definition}
\begin{definition}
\emph{Semantic duplicate} $A$ user story $\mu_1$ that duplicates the request of $\mu_2$, while using a different text; this has an impact on the unique quality criterion. Formally,\\ 
$isSemDuplicate(\mu_1,\mu_2) \leftrightarrow \mu_1 = \mu_2 \wedge \mu_1 \neq _{\text{syn}} \mu_2$
\end{definition}
\textbf{Uniform}\\ 
Uniformity pertains to the consistency of a USs format, with the majority of user stories within the same set. To evaluate uniformity, the requirements engineer identifies the most frequently occurring format, usually established in collaboration with the team. For example, the US \enquote{As an Administrator, I receive an email notification when a new user is registered} is presented as a non-uniform user story and can be rewritten for improved uniformity as: \enquote{As an Administrator, I want to receive an email notification when a new user is registered.} \\ 
\textbf{Complete}\\ 
The implementation of a set of USs should result in a feature-complete application. While it's not necessary for USs to cover 100\% of the application's functionality up-front, it's crucial not to overlook essential USs, as doing so may create a significant feature gap that hinders progress. For instance, consider the US \enquote{As a User, I am able to edit the content that I added to a person’s profile page}, which requires the existence of another story describing content creation. This scenario can be extended to USs with action verbs that reference non-existent direct objects, such as reading, updating, or deleting an item, which necessitates its creation first. To address these dependencies related to the means' direct object, Lucassen et al. introduce a conceptual relationship. 
\subsection{The Automatic Quality User Story Artisan (AQUSA)} \label{usq_4}
The QUS framework provides guidelines for improving the quality of USs. To support the framework, Lucassen et al. propose the AQUSA tool, which exposes defects and deviations from good user story practice \cite{lucassen2016improving}. AQUSA primarily targets easily describable and algorithmically determinable defects in the clerical part of requirements engineering, focusing on syntactic and some pragmatic criteria, while omitting semantic criteria that require a deep understanding of requirements' content \cite{lucassen2016improving}.
AQUSA consists of five main architectural components (Figure \ref{fig:aqusa}): linguistic parser, US base, analyzer, enhancer, and report generator.
\begin{figure}
\center
\includegraphics[width=13.03cm, height=7.76cm]{Functional_view_on_architecture_of_AQUSA}
\caption{Functional view on architecture of AQUSA. Dashed components are not fully implemented yet \cite{lucassen2016improving}}\label{fig:aqusa}
\end{figure} \\ 

The first step for every US is validating that it is well-formed. This takes place in the linguistic parser, which separates the user story in its role, means and end(s) parts. The US base captures the parsed US as an object according to the conceptual model, which acts as central storage.  Next, the analyzer runs tailormade method to verify specific syntactic and pragmatic quality criteria—where possible enhancers enrich the US base, improving the recall and precision of the analyzers. Finally, AQUSA captures the results in a comprehensive report \cite{lucassen2016improving}.

In the case of story analysis, AQUSA v1 conducts multiple analyses, beginning with the \emph{StoryChunker} and subsequently executing the Unique-, Minimal-, WellFormed-, Uniform-, and \emph{AtomicAnalyzer} modules. If any of these modules detect a violation of quality criteria, they engage the \emph{DefectGenerator} to record the defect in the associated database tables related to the story. Additionally, users have the option to utilize the AQUSA-GUI to access a project list or view a report of defects associated with a set of stories. \\ 
\textbf{Linguistic Parser: Well-Formed}\\ 
One of the essential aspects of verifying whether a string of text is a user story is splitting it into \emph{role}, \emph{means}, and \emph{end(s)}. This initial step is performed by the linguistic parser, implemented as the StoryChunker component. It identifies common indicators in the user story, such as \enquote{As a,} \enquote{I want to,} \enquote{I am able to,} and \enquote{so that.} The linguistic parser then categorizes words within each chunk using the Stanford NLP POS Tagger and validates the following rules for each chunk:
\begin{itemize}
\item Role: Checks if the last word is a noun representing an actor and if the words preceding the noun match a known role format (\emph{e.g.}, \enquote{as a}).
\item Means: Verifies if the first word is \enquote{I} and if a known means format like \enquote{want to} is present. It also ensures the remaining text contains at least one verb and one noun (\emph{e.g.}, \enquote{update event}).
\item End: Checks for the presence of an end and if it starts with a recognized end format (\emph{e.g.}, \enquote{so that}).
\end{itemize}
The linguistic parser validates whether a US adheres to the conceptual model. When it cannot detect a known means format, it retains the full user story and eliminates the role and end sections. If the remaining text contains both a verb and a noun, it's tagged as a \enquote{potential means,} and further analysis is conducted. Additionally, the parser checks for a comma after the role section. \\ 
\textbf{User Story Base and Enhancer}\\ 
Linguistically parsed USs are transformed into objects containing role, means, and ends components, aligning with the first level of decomposition in the conceptual model. These parsed USs are stored in the user story base for further processing. AQUSA enriches these USs by adding potential synonyms, homonyms, and relevant semantic information sourced from an ontology to the pertinent words within each chunk. Additionally, AQUSA includes a corrections' subpart, ensuring precise defect correction where possible. \\ 
\textbf{Analyzer: Explicit Dependencies}\\ 
AQUSA enforces that USs with explicit dependencies on other USs should include navigable links to those dependencies. It checks for numbers within USs and verifies whether these numbers are enclosed within links. For instance, if a US reads, \enquote{As a care professional, I want to edit the planned task I selected—see 908,} AQUSA suggests changing the isolated number to \enquote{See PID-908,} where PID represents the project identifier. When integrated with an issue tracker like Jira or Pivotal Tracker, this change would automatically generate a link to the dependency, such as \enquote{see PID-908 (\href{http://company.issuetracker.org/PID-908)}{http://company.issuetracker.org/PID-908}.} It's worth noting that this explicit dependency analyzer has not been implemented in AQUSA v1 to ensure its universal applicability across various issue trackers.\\ 
\textbf{Analyzer: Atomic}\\ 
AQUSA examines USs to ensure that the means section focuses on a single feature. To do this, it parses the means section for occurrences of the conjunctions \enquote{and, \&, +, or}. If AQUSA detects double feature requests in a US, it includes them in its report and suggests splitting the US into multiple ones. 
For example, a US like \enquote{As a User, I’m able to click a particular location from the map and thereby perform a search of landmarks associated with that latitude-longitude combination} would prompt a suggestion to split it into two USs: (1) \enquote{As a User, I want to click a location from the map} and (2) \enquote{As a User, I want to search landmarks associated with the lat-long combination of a location.}

AQUSA v1 verifies the role and means chunks for the presence of the conjunctions \enquote{and, \&, +, or}. If any of these conjunctions are found, AQUSA checks whether the text on both sides of the conjunction conforms to the QUS criteria for valid roles or means. Only if these criteria are met, AQUSA records the text following the conjunction as an atomicity violation. \\ 
\textbf{Analyzer: Minimal}\\ 
AQUSA assesses the minimality of USs by examining the role and means of sections extracted during chunking and \emph{well-formedness} verification. If AQUSA successfully extracts these sections, it checks for any additional text following specific punctuation marks such as dots, hyphens, semicolons, or other separators. For instance, in the US \enquote{As a care professional I want to see the registered hours of this week (split into products and activities). See: Mockup from Alice NOTE: First create the overview screen—Then add validations,} AQUSA would flag all text following the first dot (\enquote{.}) as non-minimal. Additionally, any text enclosed within parentheses is also marked as non-minimal.
AQUSA v1 employs two separate minimality checks using regular expressions. The first check searches for occurrences of special punctuation marks like \enquote{, -, ?, ., *.} and marks any text following them as a minimality violation. The second check identifies text enclosed in brackets such as \enquote{(), [], \{\}, \textless\textgreater} and records it as a minimality violation. \\ 
\textbf{Analyzer: Uniform}\\ 
AQUSA, in addition to its chunking process, identifies and extracts the format parts of USs and calculates their occurrences across all USs in a set. The most frequently occurring format is designated as the standard US format. Any US that deviates from this standard format is marked as non-compliant and included in the error report. For example, if the standard format is \enquote{I want to,} AQUSA will flag a US like \enquote{As a User, I am able to delete a landmark} as non-compliant because it does not follow the standard.
After the linguistic parser processes all USs in a set, AQUSA v1 initially identifies the most common US format by counting the occurrences of indicator phrases and selecting the most frequent one. Later, the uniformity analyzer calculates the edit distance between the format of each individual US chunk and the most common format for that chunk. If the edit distance exceeds a threshold of 3, AQUSA v1 records the entire story as a uniformity violation. This threshold ensures that minor differences, like \enquote{I am} versus \enquote{I'm,} do not trigger uniformity violations, while more significant differences in phrasing, such as \enquote{want} versus \enquote{can,} \enquote{need,} or \enquote{able,} do. \\ 
\textbf{Analyzer: Unique}\\ 
AQUSA has the capability to utilize various similarity measures, leveraging the WordNet lexical database to detect semantic similarity. For each verb and object found in the means or end of a US, AQUSA performs a WordNet::Similarity calculation with the corresponding verbs or objects from all other USs. These individual calculations are combined to produce a similarity degree for two USs. If this degree exceeds 90\%, AQUSA flags the USs as potential duplicates. \\ 
\textbf{AQUSA-GUI: report generator}\\ 
After AQUSA detects a violation in the linguistic parser or one of the analyzers, it promptly creates a defect record in the database, including details such as the defect type, a highlight of where the defect is located within the US, and its severity. AQUSA utilizes this data to generate a comprehensive report for the user.
The report begins with a dashboard that provides a quick overview of the US set's quality. It displays the total number of issues, categorized into defects and warnings, along with the count of perfect stories. Below the dashboard, all USs containing issues are listed, accompanied by their respective warnings and errors. An example is illustrated in figure \ref{fig:aqusa_report}.
\begin{figure}
\center
\includegraphics[width=11.03cm, height=9.76cm]{Example_report_of_a_defect_and_warning_for_a_story_in_AQUSA}
\caption{Example report of a defect and warning for a story in AQUSA \cite{lucassen2016improving}}\label{fig:aqusa_report}
\end{figure}
\input{Section/USQ_5}
















 






